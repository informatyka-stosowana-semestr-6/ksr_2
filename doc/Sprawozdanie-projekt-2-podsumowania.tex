\documentclass{classrep}
\usepackage[utf8]{inputenc}
\usepackage{color}

\studycycle{Informatyka, studia STACJONARNE, I st.}
\coursesemester{VI}

\coursename{Komputerowe systemy rozpoznawania}
\courseyear{2022/2023}

\courseteacher{prof. dr hab. inż. Adam Niewiadomski}
\coursegroup{Poniedziałek 13:45}

\author{
  \studentinfo{Przemysław Lis}{229940} \and
  \studentinfo{Michał Olczak}{229972} }

\title{Projekt 2.  Podsumowania lingwistyczne relacyjnych baz danych}

\begin{document}
\maketitle


\section{Cel}
Celem projektu jest stworzenie aplikacji okienkowej z dostępem do relacyjnej bazy danych, której główną funkcjonalnością jest lingwistyczna agregacja zbioru danych. Aplikacja umożliwia automatyczne generowanie podsumowań lingwistycznych służących do tworzenia krótkich wiadomości tekstowych.


\section{Baza danych, zmienne lingwistyczne, kwantyfikatory lingwistyczne}
\noindent {\bf Sekcja uzupełniona jako efekt zadania Tydzień 09 wg Harmonogramu Zajęć na WIKAMP KSR.}

\subsection{Charakterystyka podsumowywanej bazy danych}
Jako zbiór danych wybraliśmy baze danych Perth House Prices pochodząca ze strony kaggle.com [1] (https://www.kaggle.com/datasets/syuzai/perth-house-prices). Jest to zbiór mieszkań pochodzących z miasta Perth w Australii. Znajduje się tam 33656 rekordów oraz 19 kolumn z czego 13 kolumn jest postaci liczbowej oraz 6 w postaci łańcucha znaków. Do naszego eksperymentu rozpatrzymy kolejne atrybuty:
\begin{enumerate}
\item PRICE - Cena z przedziału [51000; 2440000]
\item GARAGE - Liczba garaży z przedziału [1; 99]
\item LAND AREA - powierzchnia działki w metrach kwatratowych z przedziału [61; 999999]
\item FLOOR AREA - powierzchnia podłogi w metrach kwadratowych z przedziału [1; 870]
\item BUILD YEAR - rok budowy z przedziału [1868; 2017]
\item CBD DIST - odległość do centrum z przedziału [681; 59800]
\item NEAREST STN DIST - odległość do najbliższej stacji pociągów z przedziału [46; 35500]
\item LATITUDE - wysokość geograficzna z przedziału [-32.47297865; -31.45745]
\item LONGITUDE - szerokość geograficzna z przedziału [115.58273; 116.343201]
\item NEAREST SCH DIST - odległość do najbliższej szkoły z przedziału [0.0709120036079524; 23.2543724776336]
\end{enumerate}

\subsection{Zmienne lingwistyczne (atrybuty/własności obiektów)}
Zmienne lingwistyczne dla wybranych 10 atrybutów z bazy danych, przedstawione w
formie wykresów funkcji przynależności i wzorów analitycznych.
\begin{enumerate}
	\item Cena\\
	Przestrzeń rozważań zmiennej lingwistycznej:  [51000; 2440000]

    	    
\end{enumerate}
\subsection{Kwantyfikatory lingwistyczne (liczności obiektów)}
Jw. kwantyfikatory lingwistyczne -- opisane etykietami, wykresami funkcji
przynależności i wzorami analitycznymi. Uzasadnione wiedzą dziedzinową  
{\bf zakresy i etykiety}. Precyzyjnie podane przestrzenie rozważań każdego kwantyfikatora 
lingwistycznego/rozmytego, wzory i wykresy dla każdej wartości/etykiety. Opisy własne z~przypisami do literatury, tak by inżynier innej specjalności zrozumiał dalszy
opis tego konkretnego ćwiczenia/eksperymentu.  

\section{Narzędzia obliczeniowe: wybór/implementacja. Diagram UML pakietu
obliczeń rozmytych i~generatora podsumowań. Instrukcja użytkownika}
\noindent {\bf Sekcja uzupełniona jako efekt zadania Tydzień 10 wg Harmonogramu Zajęć na WIKAMP KSR.}

Diagram UML i zwięzły opis pakietu obliczeń rozmytych: źródło pakietu
(zewnętrzny/własny/hybrydowy), przypis do literatury/źródeł. Krótka charakterystyka
najważniejszych klas i podstawowych dla zadania ich metod. \\

Diagram UML generatora podsumowań (warstwy obliczeniowej oraz interfejsu
użytkownika). Krótki ilustrowany opis jak użytkownik może korzystać z aplikacji, w~szczególności
wprowadzać parametry  podsumowań, odczytywać wyniki oraz definiować własne etykiety i
kwantyfikatory.\\

Wersja JRE i inne wymogi niezbędne do uruchomienia aplikacji przez użytkownika na własnym komputerze. 

\section{ Jednopodmiotowe podsumowania lingwistyczne. Miary jakości, podsumowanie optymalne}
Wyniki kolejnych eksperymentów wg punktów 2.-4. opisu projektu 2.  Listy podsumowań
jednopodmiotowych i tabele/rankingi podsumowań dla danych atrybutów obowiązkowe i dokładnie opisane w ,,captions'' (tytułach), konieczny opis kolumn i wierszy tabel. Dla każdego podsumowania podane miary jakości oraz miara jakości podsumowania
optymalnego. {\bf Wzorów podsumowań ani miar nie należy przepisywać ani cytować, wystarczy podać literaturę, ale
należy skomentować co oznaczają i jaką informacje niosą wybrane miary w wybranych
przypadkach.}\\
\noindent {\bf Sekcja uzupełniona jako efekt zadania Tydzień 11 wg Harmonogramu Zajęć na WIKAMP KSR.}

\section{Wielopodmiotowe podsumowania lingwistyczne i~ich miary jakości} 
Wyniki kolejnych eksperymentów wg punktów 2.-4. opisu projektu 2. Uzasadnienie i
metoda podziału zbioru danych na rozłączne podmioty. Listy podsumowań
wielopodmiotowych i tabele/rankingi podsumowań dla danych atrybutów obowiązkowe i
dokładnie opisane w ,,captions'' (tytułach), konieczny opis kolumn i wierszy tabel.
{\bf Wzorów podsumowań ani miar nie należy przepisywać ani cytować, wystarczy podać literaturę, ale
należy skomentować co oznaczają i jaką informacje niosą wybrane miary w wybranych
przypadkach.}Konieczne uwzględnienie wszystkich 4-ch form podsumowań wielopodmiotowych. 
\\ 

** Możliwe sformułowanie zagadnienia wielopodmiotowego podsumowania optymalnego **.\\
\indent {** Ewentualne wyniki realizacji punktu ,,na ocenę 5.0'' wg opisu Projektu 2. i ich porównanie do wyników z
części obowiązkowej **.}\\

\noindent {\bf Sekcja uzupełniona jako efekt zadania Tydzień 12 wg Harmonogramu Zajęć
na WIKAMP KSR.}


\section{Dyskusja, wnioski}
Dokładne interpretacje uzyskanych wyników w zależności od parametrów klasyfikacji
opisanych w punktach 3.-4 opisu Projektu 2. 
Szczególnie istotne są wnioski o charakterze uniwersalnym, istotne dla podobnych zadań. 
Omówić i wyjaśnić napotkane problemy (jeśli były). Każdy wniosek/problem powinien mieć poparcie
w przeprowadzonych eksperymentach (odwołania do konkretnych wyników: tabel i miar
jakości). Ocena które wybrane kwantyfikatory, sumaryzatory, kwalifikatory i/lub ich
miary jakości mają małe albo duże znaczenie dla wiarygodności i jakości otrzymanych
agregacji/podsumowań.  \\
\underline{Dla końcowej oceny jest to najważniejsza sekcja} sprawozdania, gdyż prezentuje poziom
zrozumienia rozwiązywanego problemu.\\

** Możliwości kontynuacji prac w obszarze logiki rozmytej i wnioskowania rozmytego, zwłaszcza w kontekście pracy inżynierskiej,
magisterskiej, naukowej, itp. **\\

\noindent {\bf Sekcja uzupełniona jako efekt zadań Tydzień 11 i Tydzień 12 wg
Harmonogramu Zajęć na WIKAMP KSR.}


\section{Braki w realizacji projektu 2.}
Wymienić wg opisu Projektu 2. wszystkie niezrealizowane obowiązkowe elementy projektu, ewentualnie
podać merytoryczne (ale nie czasowe) przyczyny tych braków. 


\begin{thebibliography}{99}
 \bibitem{niewiadomski19} A. Niewiadomski, Zbiory rozmyte typu 2. Zastosowania w reprezentowaniu informacji.  Seria „Problemy współczesnej informatyki” pod redakcją L. Rutkowskiego. Akademicka Oficyna Wydawnicza EXIT, Warszawa, 2019.
\bibitem{zadrozny06} S. Zadrożny, Zapytania nieprecyzyjne i lingwistyczne podsumowania baz danych, EXIT, 2006, Warszawa
\bibitem{niewiadomski08} A. Niewiadomski, Methods for the Linguistic Summarization of Data: Applications of Fuzzy Sets and Their Extensions, Akademicka Oficyna Wydawnicza EXIT, Warszawa, 2008.
\end{thebibliography}

Literatura zawiera wyłącznie źródła recenzowane i/lub o potwierdzonej wiarygodności,
możliwe do weryfikacji i cytowane w sprawozdaniu. 
\end{document}
